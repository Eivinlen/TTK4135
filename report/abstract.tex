\begin{abstract}\addcontentsline{toc}{section}{Abstract}
%First re-read your paper/report for an overview. Then read each section and condense the information in each down to 1-2 sentences.
% - Next read these sentences again to ensure that they cover the major points in your paper.
% - Ensure you have written something for each of the key points outlined above for either the descriptive or informative abstract.
% - Check the word length and further reduce your words if necessary by cutting out unnecessary words or rewriting some of the sentences into a single, more succinct sentence.
% - Edit for flow and expression.

The purpose of the helicopter project in TTK4135 is to get experience combining optimization algorithm and control theory on a real world platform. The helicopter will be controlled by the optimal trajectory / input according to a linearized model, both in open-loop and closed-loop with a linear quadratic regulator. The project also gives an introduction to hardware-in-the-loop (HIL) and automatic generation of code using MatLab with QuaRC.

The main topics of this project is:
\begin{itemize}
\item Brief introduction to the model of the helicopter and the assumptions done when linearizing it.
\item Discretization of the model
\item Optimal control with cost function and linear inequality constraints
 \item Applying linear quadratic regulator (LQR) to the optimal trajectory
 \item Optimal control in two dimensions with non-linear inequality constraints
\end{itemize}


\end{abstract}
