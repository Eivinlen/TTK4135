\section{Optimal Control of Pitch/Travel and Elevation with and without Feedback}\label{sec:prob4}
In this section we extend our model to include the remaining states: elevation, $e$, and elevation
rate, $\dot{e}$. We use a non-linear solver to compute an optimal trajectory in two dimensions,
and additionally constrain the elevation to avoid a restriction shaped as a bell-curve.

\subsection{State-space formulation}
The state-vector is extended with the remaining states, $x = \begin{bmatrix} \lambda & r & p & \dot{p} & e & \dot{e} \end{bmatrix}^T$, and the input-vector also contains the elevation setpoint fed to the internal controller, $u = \begin{bmatrix} p_c & e_c \end{bmatrix}^T$. The system is on the usual state-space form (\ref{eq:state_space_axbu}),
with
\begin{equation}
    \dot{x} =
    \underbrace{
    \begin{bmatrix}
    0 & 1 &      0     &      0     &      0     &      0    \\
    0 & 0 &    -K_2    &      0     &      0     &      0    \\
    0 & 0 &      0     &      1     &      0     &      0    \\
    0 & 0 & -K_1K_{pp} & -K_1K_{pd} &      0     &      0    \\
    0 & 0 &      0     &      0     &      0     &      1    \\
    0 & 0 &      0     &      0     & -K_3K_{ep} & -K_3K_{ed}
    \end{bmatrix}}_{A_c}
    x +
    \underbrace{
    \begin{bmatrix}
        0       &     0     \\
        0       &     0     \\
        0       &     0     \\
    K_1K_{pp}   &     0     \\
        0       &     0     \\
        0       & K_3K_{ep}
    \end{bmatrix}}_{B_c}
    u
    \label{eq:extended_state_space}
\end{equation}

\subsection{Discretization}
We discretize (\ref{eq:extended_state_space}) using the same method
as in section (\ref{sec:prob2}). That is, an approximation of the
discrete-time state-space matrices is
\begin{equation}
    A \approx I + hA_c
    \qquad\text{and}\qquad
    B \approx hB_c
\end{equation}
where $I$ is now the $6\times6$ identity matrix.

\subsection{Modelling the restriction}
A common application of optimal control is to implement restrictions,
such as avoiding physical objects, as constraints in the optimization
problem. Such restrictions can not be enforced when using only state-
feedback controllers.

We wish to restrict the helicopter head to move above a bell-shaped curve
\begin{equation}
    e_k \geq \alpha \exp (-\beta (\lambda_k - \lambda_t)^2 )
\end{equation}
for all timesteps $k$ over the solution horizon. Since this is a non-
linear constraint, we can no longer use a QP solver.

% \section{Notater for dagen: 16. mars}
% Fikk fmincon til å fungere, med ikke-lineær constraint.
% TODO: Test uten LQR. Lag egen slx hvor vi kopler ut LQR blokka.

% TODO: Kjør tuning tester. Juster Q_LQR.

\subsection{Results}

% TODO: Add figure showing the helicopter following the non-linear constraint. Showing the associated input. Explain why the input looks like it does.

% Notation:
% x_e^*: Computed elevation trajectory
% x_t^*: Computed travel trajectory
% u_p^*: Computed pitch setpoint sequence
% u_e^*: Computed elevation setpoint sequence


% Notat
% Cirka 11 sek ut i simuleringa, så ser vi at det er avvik mellom humpen og faktisk elevation, samtidig som det er avvik mellom travel-bane og faktisk travel. Vi har forsøkt å tune slik at elevation følger humpen tettere, men uten hell.

% Grunnen til dette tror vi er fordi det er en tradeoff i virkeligheten mellom å følge travel-banen og humpen. For å følge humpen bedre, så brukes pitch på en slik måte at det blir større avvik fra travel-banen. I modellen antas at pitch og elevation er fullstendig separerte, og dette problemet skal i teorien ikke oppstå. Men i virkeligheten er det en kobling.

% TODO: Forklar dette fenomenet. Referér til figuren, ved hump-toppen.
