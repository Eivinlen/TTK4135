\section{Intro}
\subsection{Modeling, linearizion}
On this project we will be using optimization theory to control a model of a helicopter. This is done by making a non-linear model of the helicopter, and the linearize it at an equilibrium. This method is widely used in control theory and works well if the system operates near or at the equilibrium or have quite small non-linearities. However when this is not the case, a linearized model may not be adequate for stabilizing and controlling the system. The reason most control engineers prefer linear model is the ability to analyze them, and tune regulators. In this project we linearize one time manually around an equilibrium, but there is no limitations on the number of times the model can be linearized and the process can be done automatically by a computer. More about this in the last section.

\subsection{Optimal control, limitations without feedback}
Optimization theory is a knowledge used in many applications, from economy and production planing to control theory. In this project we will be looking at the last application, and how it can be used to solve control theory problems. The main difference between a conventional regulator who only uses the current state or output and a reference and a trajectory from an optimization over a horizon is the ability to "look into the future". While a conventional regulator will calculate an error and correct the input thereafter, a planned trajectory from an optimization problem will contain inputs for the complete horizon. However there can be quite drastic drawbacks with implementing this optimal sequence of inputs directly to the system since there are no feedback, any error caused by modeling errors and disturbances will accumulate and often cause drift and maybe instability.

\subsection{Optimal control, trust the trajectory not the input}
However, while the optimal sequence of inputs may not be perfect in open-loop, the planned trajectory for the states are often quite useful, even if the model have errors. Here it is possible to implement a traditional PID regulator, or as we will be doing in this project, implement a linear quadratic regulator (LQR). A LQR can be both time variant and time invariant, dictated by the nature of the system. As mentioned in the linearizion part, the calculation of the LQR gain can and will be done by a computer, and will often be recalculated when re-linearizing. By tuning the LQR so that it prioritize to follow the trajectory of the states more than following the optimal input, it is possible to compensate for modeling errors.

\subsection{Optimal control with non-linear constraints}
When controlling two degrees of freedom with non-linear inequality constraints with optimization theory, the time it takes to solve the optimization problem is quite long. When calculating the trajectory once and then executing the whole horizon this does not become a problem.

\subsection{MPC with enough computation power}
 However when using MPC, which is to recalculate the horizon every time step and only use the first calculated input, this time becomes a problem. This calculation time can be improved by selecting a better solver, increase hardware specifications and do simplifications / linearizion. When achieving this on a low-power unit like a micro controller, FPGA, DSP or equivalent, a new world in control theory will open.