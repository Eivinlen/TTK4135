\section{Problem Description}\label{sec:prob_descr}
In this section you should describe the lab setup and discuss the model. If you want, you can copy the source code for the model equations:
\begin{gather}
	\ddot{e} + K_{3} K_{ed} \dot{e} + K_{3} K_{ep} e = K_{3} K_{ep} e_{c} \label{eq:model_elev} \\
	\ddot{p} + K_{1} K_{pd} \dot{p} + K_{1} K_{pp} p = K_{1} K_{pp} p_{c} \label{eq:model_pitch} \\
	\dot{\lambda} = r \label{eq:model_lambda} \\
	\dot{r} = -K_{2} p \label{eq:model_r} 
\end{gather}
Since these equations belong together, it's a good idea to number them like this:
\begin{subequations}
\label{eq:model}
\begin{gather}
	\ddot{e} + K_{3} K_{ed} \dot{e} + K_{3} K_{ep} e = K_{3} K_{ep} e_{c} \label{eq:model_se_elev} \\
	\ddot{p} + K_{1} K_{pd} \dot{p} + K_{1} K_{pp} p = K_{1} K_{pp} p_{c} \label{eq:model_se_pitch} \\
	\dot{\lambda} = r \label{eq:model_se_lambda} \\
	\dot{r} = -K_{2} p \label{eq:model_se_r} 
\end{gather}
\end{subequations}
You can then both reference individual equations (``the elevation equation \eqref{eq:model_se_elev}'') or reference the entire model (``the linear model \eqref{eq:model}''). Regardless of your choice of software, never hard-code a reference, always use dynamic references. 

You could also align the equations like this:
\begin{subequations}
\label{eq:model_al}
\begin{align}
	\ddot{e} + K_{3} K_{ed} \dot{e} + K_{3} K_{ep} e &= K_{3} K_{ep} e_{c} \label{eq:model_se_al_elev} \\
	\ddot{p} + K_{1} K_{pd} \dot{p} + K_{1} K_{pp} p &= K_{1} K_{pp} p_{c} \label{eq:model_se_al_pitch} \\
	\dot{\lambda} &= r \label{eq:model_se_al_lambda} \\
	\dot{r} &= -K_{2} p \label{eq:model_se_al_r} 
\end{align}
\end{subequations}
You can consult \citet{Downes2002} for more about writing math.

If you decide to include a figure, that's great. You can in general copy figures from the textbook, the assignement text, or other places. However: ALWAYS CITE THE SOURCE.